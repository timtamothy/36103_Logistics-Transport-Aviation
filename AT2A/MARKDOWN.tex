% Options for packages loaded elsewhere
\PassOptionsToPackage{unicode}{hyperref}
\PassOptionsToPackage{hyphens}{url}
%
\documentclass[
]{article}
\usepackage{amsmath,amssymb}
\usepackage{lmodern}
\usepackage{ifxetex,ifluatex}
\ifnum 0\ifxetex 1\fi\ifluatex 1\fi=0 % if pdftex
  \usepackage[T1]{fontenc}
  \usepackage[utf8]{inputenc}
  \usepackage{textcomp} % provide euro and other symbols
\else % if luatex or xetex
  \usepackage{unicode-math}
  \defaultfontfeatures{Scale=MatchLowercase}
  \defaultfontfeatures[\rmfamily]{Ligatures=TeX,Scale=1}
\fi
% Use upquote if available, for straight quotes in verbatim environments
\IfFileExists{upquote.sty}{\usepackage{upquote}}{}
\IfFileExists{microtype.sty}{% use microtype if available
  \usepackage[]{microtype}
  \UseMicrotypeSet[protrusion]{basicmath} % disable protrusion for tt fonts
}{}
\makeatletter
\@ifundefined{KOMAClassName}{% if non-KOMA class
  \IfFileExists{parskip.sty}{%
    \usepackage{parskip}
  }{% else
    \setlength{\parindent}{0pt}
    \setlength{\parskip}{6pt plus 2pt minus 1pt}}
}{% if KOMA class
  \KOMAoptions{parskip=half}}
\makeatother
\usepackage{xcolor}
\IfFileExists{xurl.sty}{\usepackage{xurl}}{} % add URL line breaks if available
\IfFileExists{bookmark.sty}{\usepackage{bookmark}}{\usepackage{hyperref}}
\hypersetup{
  pdftitle={Airline Delay Prediction Proposal},
  pdfauthor={Logistics Transport Aviation},
  hidelinks,
  pdfcreator={LaTeX via pandoc}}
\urlstyle{same} % disable monospaced font for URLs
\usepackage[margin=1in]{geometry}
\usepackage{color}
\usepackage{fancyvrb}
\newcommand{\VerbBar}{|}
\newcommand{\VERB}{\Verb[commandchars=\\\{\}]}
\DefineVerbatimEnvironment{Highlighting}{Verbatim}{commandchars=\\\{\}}
% Add ',fontsize=\small' for more characters per line
\usepackage{framed}
\definecolor{shadecolor}{RGB}{248,248,248}
\newenvironment{Shaded}{\begin{snugshade}}{\end{snugshade}}
\newcommand{\AlertTok}[1]{\textcolor[rgb]{0.94,0.16,0.16}{#1}}
\newcommand{\AnnotationTok}[1]{\textcolor[rgb]{0.56,0.35,0.01}{\textbf{\textit{#1}}}}
\newcommand{\AttributeTok}[1]{\textcolor[rgb]{0.77,0.63,0.00}{#1}}
\newcommand{\BaseNTok}[1]{\textcolor[rgb]{0.00,0.00,0.81}{#1}}
\newcommand{\BuiltInTok}[1]{#1}
\newcommand{\CharTok}[1]{\textcolor[rgb]{0.31,0.60,0.02}{#1}}
\newcommand{\CommentTok}[1]{\textcolor[rgb]{0.56,0.35,0.01}{\textit{#1}}}
\newcommand{\CommentVarTok}[1]{\textcolor[rgb]{0.56,0.35,0.01}{\textbf{\textit{#1}}}}
\newcommand{\ConstantTok}[1]{\textcolor[rgb]{0.00,0.00,0.00}{#1}}
\newcommand{\ControlFlowTok}[1]{\textcolor[rgb]{0.13,0.29,0.53}{\textbf{#1}}}
\newcommand{\DataTypeTok}[1]{\textcolor[rgb]{0.13,0.29,0.53}{#1}}
\newcommand{\DecValTok}[1]{\textcolor[rgb]{0.00,0.00,0.81}{#1}}
\newcommand{\DocumentationTok}[1]{\textcolor[rgb]{0.56,0.35,0.01}{\textbf{\textit{#1}}}}
\newcommand{\ErrorTok}[1]{\textcolor[rgb]{0.64,0.00,0.00}{\textbf{#1}}}
\newcommand{\ExtensionTok}[1]{#1}
\newcommand{\FloatTok}[1]{\textcolor[rgb]{0.00,0.00,0.81}{#1}}
\newcommand{\FunctionTok}[1]{\textcolor[rgb]{0.00,0.00,0.00}{#1}}
\newcommand{\ImportTok}[1]{#1}
\newcommand{\InformationTok}[1]{\textcolor[rgb]{0.56,0.35,0.01}{\textbf{\textit{#1}}}}
\newcommand{\KeywordTok}[1]{\textcolor[rgb]{0.13,0.29,0.53}{\textbf{#1}}}
\newcommand{\NormalTok}[1]{#1}
\newcommand{\OperatorTok}[1]{\textcolor[rgb]{0.81,0.36,0.00}{\textbf{#1}}}
\newcommand{\OtherTok}[1]{\textcolor[rgb]{0.56,0.35,0.01}{#1}}
\newcommand{\PreprocessorTok}[1]{\textcolor[rgb]{0.56,0.35,0.01}{\textit{#1}}}
\newcommand{\RegionMarkerTok}[1]{#1}
\newcommand{\SpecialCharTok}[1]{\textcolor[rgb]{0.00,0.00,0.00}{#1}}
\newcommand{\SpecialStringTok}[1]{\textcolor[rgb]{0.31,0.60,0.02}{#1}}
\newcommand{\StringTok}[1]{\textcolor[rgb]{0.31,0.60,0.02}{#1}}
\newcommand{\VariableTok}[1]{\textcolor[rgb]{0.00,0.00,0.00}{#1}}
\newcommand{\VerbatimStringTok}[1]{\textcolor[rgb]{0.31,0.60,0.02}{#1}}
\newcommand{\WarningTok}[1]{\textcolor[rgb]{0.56,0.35,0.01}{\textbf{\textit{#1}}}}
\usepackage{graphicx}
\makeatletter
\def\maxwidth{\ifdim\Gin@nat@width>\linewidth\linewidth\else\Gin@nat@width\fi}
\def\maxheight{\ifdim\Gin@nat@height>\textheight\textheight\else\Gin@nat@height\fi}
\makeatother
% Scale images if necessary, so that they will not overflow the page
% margins by default, and it is still possible to overwrite the defaults
% using explicit options in \includegraphics[width, height, ...]{}
\setkeys{Gin}{width=\maxwidth,height=\maxheight,keepaspectratio}
% Set default figure placement to htbp
\makeatletter
\def\fps@figure{htbp}
\makeatother
\setlength{\emergencystretch}{3em} % prevent overfull lines
\providecommand{\tightlist}{%
  \setlength{\itemsep}{0pt}\setlength{\parskip}{0pt}}
\setcounter{secnumdepth}{-\maxdimen} % remove section numbering
\usepackage{booktabs}
\usepackage{longtable}
\usepackage{array}
\usepackage{multirow}
\usepackage{wrapfig}
\usepackage{float}
\usepackage{colortbl}
\usepackage{pdflscape}
\usepackage{tabu}
\usepackage{threeparttable}
\usepackage{threeparttablex}
\usepackage[normalem]{ulem}
\usepackage{makecell}
\usepackage{xcolor}
\ifluatex
  \usepackage{selnolig}  % disable illegal ligatures
\fi

\title{Airline Delay Prediction Proposal}
\usepackage{etoolbox}
\makeatletter
\providecommand{\subtitle}[1]{% add subtitle to \maketitle
  \apptocmd{\@title}{\par {\large #1 \par}}{}{}
}
\makeatother
\subtitle{STDS - AT2}
\author{Logistics Transport Aviation}
\date{30/08/2021}

\begin{document}
\maketitle

{
\setcounter{tocdepth}{3}
\tableofcontents
}
\hypertarget{airline-delay-prediction}{%
\subsection{Airline Delay Prediction}\label{airline-delay-prediction}}

Aircraft delays have serious economic impacts that represent a
logistical headache for airlines, airports, airline crew, and passengers
alike. In a 2010 study by the National Center of Excellence For Aviation
Operations Research, an estimated 31.2 billion US dollars was lost due
to direct and indirect costs of airline delays in the United States for
2007. These expenses included maintenance costs, extra paid hours for
flight crew, the extra use of fuel, and airport fees. As an example,
airlines have dedicated time slots at airport gates and unexpected
delays exceeding the expected time slot can cost thousands of dollars by
the minute.

Many factors can contribute to aircraft delays. These may include
unavoidable reasons such as inclement weather, unforeseen aircraft
mechanical breakdowns, or a multitude of passenger-related incidents.
Airline delays then create complicated scenarios for airline schedulers
as a delay for one aircraft may impact the flight schedule for all
consecutive flights using that aircraft. Passengers with connecting
flights may need to be ticketed on other flights, reserve flight crew
may need to be called if the current flight crew are going to exceed
their flight hours, and delays may continue to have a snowball effect on
later flights causing them to also be delayed (Ball, 2010).

Thus, understanding the impact of these variables on aircraft on-time
performance can help elucidate ways and methods to mitigate these
airline delays. By doing so, steps can be made to reduce airline costs
associated with these delays, benefiting the airlines' stakeholders. We
identify the primary stakeholders to be investors and shareholders for
the financial performance of the airline. In addition, airline
passengers are important stakeholders as delays negatively influence
repurchase intentions and sentiment across word-of-mouth (Kim, 2016).

Herein, we propose an exploratory analysis and prediction model on
airline delays utilizing data from the United States in 2019.

\hypertarget{stakeholders}{%
\subsection{Stakeholders}\label{stakeholders}}

\begin{table}
\centering
\begin{tabular}[t]{>{}l||>{\raggedright\arraybackslash}p{30em}}
\hline
Stakeholders & Description\\
\hline
Airline company executives & The main participants in the aviation industry. 
\textbf{This project can help them improve their operational planning and execution and any related products and services.}\\
\hline
\textbf{Airline company employees} & Pilots, engineers, and flight attendants ensure that passengers' needs are appropriately met and satisfied.\\
\hline
\textbf{Shareholders and investors of Airline companies} & Their primary role is to provide capital in the form of share capital. Airlines performance may affect their investing behaviours.\\
\hline
\textbf{Consumers of airline companies} & Airline company/industry’s primary revenue source. Customers are the most important factor in this industry; without customers, airlines would not exist.\\
\hline
\textbf{Suppliers for Airline companies} & Form in which how much the company orders supplies in impacts inventory, e.g Aircraft manufacters, fuel and food/beverage resources.\\
\hline
\textbf{Government} & Government and policy makers can impact how the industry will be regulated, such as climate change, tourism, and infrastructure policy.\\
\hline
\end{tabular}
\end{table}

\hypertarget{research-questions}{%
\subsection{Research Questions}\label{research-questions}}

As the airline network contains complex interactions between regulatory,
passenger, and airline-specific needs, predicting delays and
understanding the primary causes is relatably complicated.

Our main question to answer is: \textbf{Which factors are most
correlated to, and influence greatly, on-time aircraft performance?}.

From here, smaller questions can be asked such as: Are delays impacted
by weather, date, time of day, route, aircraft type, or economic
category?

\hypertarget{datasets}{%
\subsection{Datasets}\label{datasets}}

The main dataset is acquired through The United States Department of
Transportation. It is a comprehensive dataset with delay times, airline
codes, origin and destination airport codes, dates, flight time, etc.
This dataset contains 109 variables classified as:

\begin{itemize}
\tightlist
\item
  6 Time period
\item
  5 Airline variables
\item
  9 variables of origin
\item
  9 variables of destination
\item
  9 departure performance variables
\item
  9 arrival performance variables
\item
  62 additional variables including information about cancellation,
  diversions, causes of delay, and summaries.
\end{itemize}

To improve the readability of the dataset, we included a lookup table
that has information on airline codes and their airline names.
Furthermore, the airport name corresponding to airport codes are
retrieved through an API, then merged with the main dataset. More
datasets, such as Skytrax Airline review which record passenger
sentiment will be considered and might be used for merging in later
stages. Other examined datasets included US weather events, airplane
crash and fatality records, aircraft employment data, and records for
airlines and their aircraft types.

After reviewing the datasets, we can say that the granularity of the
data sources is sufficient to answer our research questions (see above).
Our main data set is obtained through a reliable source (US DOT), has a
really high usability and contains a great amount of variables from
different categories which makes it easier for us to merge with
different datasets, explore our research questions deeply and find
insights.

\hypertarget{regression-modelling-techniques}{%
\subsection{Regression modelling
techniques}\label{regression-modelling-techniques}}

The regression model we are considering using in this report is a
multivariate linear regression (MLR) model. As flight delays could have
a multitude of causes, including multiple reasons per flight, a MLR
would be most suitable for this model. Some causes could be weather,
operating issues, air traffic, and seasonal variabilities such as
holidays. We will intend to examine how much influence each variable has
on the dependent variable.

For this model, we assume that there is a linear relationship between
airline delay and the various potential causes. We also assume that the
residuals are normally distributed, and that the independent variables
do not correlate with each other. We also assume that the variance of
error terms across all delays are equally distributed.

\hypertarget{impact-analysis}{%
\subsection{Impact analysis}\label{impact-analysis}}

There are few potential issues that may arise in this project. Although
the use of a web API carries the most up-to-date data, a loss of
connection with the API server could make compiling data difficult.
Secondly, the data itself contains millions of observations from 2019
through 2021. Computing power and memory storage are potential hurdles
for analysis. In fact, one dataset was over 80GB and was unable to be
loaded into the analysis program. Other potential issues may arise with
merging data where a joining dataset may be missing values. Lastly, the
statistical analysis relies on several assumptions specific to
multivariate linear regression. Should these assumptions be broken, it
may lead to difficulty in model accuracy.

\hypertarget{conclusionsummary}{%
\subsection{Conclusion/Summary}\label{conclusionsummary}}

We are proposing an in-depth analysis in airline delays in effort to
uncover possible causes for air travel's biggest headaches. This project
aims to model delays as a function of weather, airport, and airline
operations. This understanding will help airlines and customers save
time, money, and most importantly, frustration.

\hypertarget{references}{%
\subsection{References}\label{references}}

Ball, M. (2010, November). Total Delay Impact Study A Comprehensive
Assessment of the Costs and Impacts of Flight Delay in the United
States.
\url{https://isr.umd.edu/NEXTOR/pubs/TDI_Report_Final_11_03_10.pdf}

Kim, N. Y., \& Park, J. W. (2016). A study on the impact of airline
service delays on emotional reactions and customer behavior. Journal of
Air Transport Management, 57, 19--25.
\url{https://doi.org/10.1016/j.jairtraman.2016.07.005}

\hypertarget{appendix}{%
\subsection{Appendix}\label{appendix}}

\begin{Shaded}
\begin{Highlighting}[]
  \FunctionTok{library}\NormalTok{(psych)}
  \FunctionTok{library}\NormalTok{(devtools)}

  \FunctionTok{library}\NormalTok{(tidyverse)}
  \FunctionTok{library}\NormalTok{(dplyr)}
  \FunctionTok{library}\NormalTok{(ggplot2)}
  \FunctionTok{library}\NormalTok{(Amelia)}

  \FunctionTok{library}\NormalTok{(knitr)}
\end{Highlighting}
\end{Shaded}

To get the latest content, we are using a web API call for Airport
Names. The \texttt{httr::get} will assist us in fetching the content in
the JSON format. API key is being used for authenticating our call with
the API provider. Since JSON cannot be directly consumed by regression
packages, we deploy \texttt{jsonlite::fromJSON} to convert to an R
dataframe.

\begin{Shaded}
\begin{Highlighting}[]
\FunctionTok{library}\NormalTok{(httr)}
\FunctionTok{library}\NormalTok{(jsonlite)}
\FunctionTok{library}\NormalTok{(dplyr)}

\NormalTok{icao }\OtherTok{\textless{}{-}} \FunctionTok{GET}\NormalTok{(}
  \AttributeTok{url =} \StringTok{\textquotesingle{}https://applications.icao.int/dataservices/api/safety{-}characteristics{-}list?api\_key=8d00ef90{-}0982{-}11ec{-}9d72{-}8160549d64ab\&airports=\&states=USA\textquotesingle{}}
\NormalTok{)}
\NormalTok{response }\OtherTok{\textless{}{-}} \FunctionTok{content}\NormalTok{(icao, }\StringTok{\textquotesingle{}parsed\textquotesingle{}}\NormalTok{)}
\NormalTok{API\_data }\OtherTok{\textless{}{-}} \FunctionTok{fromJSON}\NormalTok{(response)}

\CommentTok{\#Only pick those columns that are useful later on }
\NormalTok{API\_data }\OtherTok{\textless{}{-}}\NormalTok{ API\_data }\SpecialCharTok{\%\textgreater{}\%} \FunctionTok{select}\NormalTok{(airportCode, airportName)}
\end{Highlighting}
\end{Shaded}

Airline On-time data

\begin{Shaded}
\begin{Highlighting}[]
\NormalTok{ontime\_raw }\OtherTok{\textless{}{-}} \FunctionTok{read.csv}\NormalTok{(}\StringTok{"https://raw.githubusercontent.com/charlottetse33/36103\_Logistics{-}Transport{-}Aviation/main/test\%20on\%20time\%20report.csv"}\NormalTok{)}
\end{Highlighting}
\end{Shaded}

Airline DOT\_ID corresponding to Airline Names

\begin{Shaded}
\begin{Highlighting}[]
\CommentTok{\#Seperate loading data to prevent R from crashing}
\NormalTok{id\_airline }\OtherTok{\textless{}{-}} \FunctionTok{read.csv}\NormalTok{(}\StringTok{"https://raw.githubusercontent.com/charlottetse33/36103\_Logistics{-}Transport{-}Aviation/main/dot\_id\_airline.csv"}\NormalTok{)}
\end{Highlighting}
\end{Shaded}

Merge Airline Names with On-time Dataset

\begin{Shaded}
\begin{Highlighting}[]
\FunctionTok{colnames}\NormalTok{(id\_airline) }\OtherTok{\textless{}{-}} \FunctionTok{c}\NormalTok{(}\StringTok{"DOT\_ID\_Reporting\_Airline"}\NormalTok{, }\StringTok{"Airline"}\NormalTok{)}
  
\NormalTok{ontime\_1 }\OtherTok{\textless{}{-}} \FunctionTok{left\_join}\NormalTok{(ontime\_raw, id\_airline, }\AttributeTok{by =} \StringTok{"DOT\_ID\_Reporting\_Airline"}\NormalTok{)}
\end{Highlighting}
\end{Shaded}

Merge API Airport names with On-time Dataset

\begin{Shaded}
\begin{Highlighting}[]
\NormalTok{ontime\_2 }\OtherTok{\textless{}{-}}\NormalTok{ ontime\_1 }\SpecialCharTok{\%\textgreater{}\%} \FunctionTok{mutate}\NormalTok{(}\AttributeTok{Origin\_airportCode =} \FunctionTok{paste}\NormalTok{(}\StringTok{"K"}\NormalTok{,Origin, }\AttributeTok{sep =} \StringTok{""}\NormalTok{), }\AttributeTok{Dest\_airportCode =} \FunctionTok{paste}\NormalTok{(}\StringTok{"K"}\NormalTok{,Dest, }\AttributeTok{sep =} \StringTok{""}\NormalTok{))}

\NormalTok{Originjoint }\OtherTok{\textless{}{-}} \FunctionTok{left\_join}\NormalTok{(ontime\_2, API\_data, }\AttributeTok{by =} \FunctionTok{c}\NormalTok{(}\StringTok{"Origin\_airportCode"} \OtherTok{=} \StringTok{"airportCode"}\NormalTok{))}
\FunctionTok{names}\NormalTok{(Originjoint)[}\FunctionTok{names}\NormalTok{(Originjoint) }\SpecialCharTok{==} \StringTok{"airportName"}\NormalTok{] }\OtherTok{\textless{}{-}} \StringTok{"Origin\_airportName"}

\NormalTok{ontime\_final }\OtherTok{\textless{}{-}} \FunctionTok{left\_join}\NormalTok{(Originjoint, API\_data, }\AttributeTok{by =} \FunctionTok{c}\NormalTok{(}\StringTok{"Dest\_airportCode"} \OtherTok{=} \StringTok{"airportCode"}\NormalTok{))}
\FunctionTok{names}\NormalTok{(ontime\_final)[}\FunctionTok{names}\NormalTok{(ontime\_final) }\SpecialCharTok{==} \StringTok{"airportName"}\NormalTok{] }\OtherTok{\textless{}{-}} \StringTok{"Dest\_airportName"}

\CommentTok{\#New column names from merging data: }
\CommentTok{\#Airline, Origin\_airportName, Dest\_airportName}
\end{Highlighting}
\end{Shaded}


\end{document}
